\cvsection{Work Experience}

\begin{cventries}

  \cventry
    {Software Engineer}
    {Waymo}
    {Mountain View, California, USA}
    {Jun 2022 - Present}
    {
      \begin{cvitems}
        \item Working on Waymo's most advanced simulation system for autonomous
          driving.
        \item Built, deployed and maintained a Machine Learning model that 
          improves the quality of the simulation by generating realistic
          pedestrian characters. Used state-of-the-art Diffusion Models trained
          on large-scale datasets of real-world pedestrians.
        \item Building and improving Large Language Model-based Transformer
          Machine Learning models to improve the quality and realism of the
          Simulation.
        \item Migrated a critical Neural Network-based model from CPU to TPU and
          reduced inference latency by 70x.
        \item Built distributed systems for the efficient data processing in
          simulation that are working on thousands of distributed nodes.
      \end{cvitems}
    }

  \cventry
    {Software Engineer}
    {Google}
    {}
    {Nov 2019 - Jun 2022}
    {
      \begin{cvitems}
        \item Significantly improved \link{clangd}
	  {https://clangd.llvm.org/features}, Clang-based C++ Language
	  Server Protocol implementation that brings IDE-like experience for
	  for Visual Studio Code, Vim, CLion, XCode, a number of
	  other IDEs and text editors.
        \item Designed and implemented \link{Include Cleaner}{https://clangd.llvm.org/design/include-cleaner}:
	  \link{IWYU}{https://include-what-you-use.org/}-like functionality
          for warning users about unused and missing headers.
        \item Implemented \link{Remote Index
          service}{https://clangd-index.llvm.org/} for LLVM and Chromium. This
          allows developers with slow machines to get precise code completion
          and navigation with zero startup time. Deployed the service to Google
	  Cloud and maintained it. The service is handling 1.5M+ requests
          per week.
        \item Took responsibilities of Product Manager and UX Researcher:
          continuously interviewed our users and processed feedback, built
          feature roadmaps and prioritized tasks based on the needs of our
          users.
      \end{cvitems}
    }

  \cventry
    {Data Scientist}
    {\link{Handl}{https://handl.ai/},
     \link{Y Combinator}{https://www.ycombinator.com/}-backed Machine Learning
     startup}
    {}
    {Feb 2019 - Jun 2019}
    {
      \begin{cvitems}
        \item Secured a contract with AMG Mercedes Benz, worked out how to
          translate their business goals into fine-grained technical problems
          and led the project.
        \item Built Deep Learning models for a variety of tasks such as Image
	  and Sound Segmentation, Image and Video Classification.
        \item Contributed to state-of-the-art Deep Learning models for Optical
          Character Recognition in PyTorch and TensorFlow.
      \end{cvitems}
    }

  \cventry
    {Software Engineering Intern}
    {Google}
    {}
    {Jul 2018 - Sep 2018}
    {
      \begin{cvitems}
        \item Designed and implemented
          \link{Dex}{https://docs.google.com/document/d/1C-A6PGT6TynyaX4PXyExNMiGmJ2jL1UwV91Kyx11gOI/edit?usp=sharing}
          --- search index for efficient code completion.
        \item Replaced previous index implementation and gained
          \textbf{15x-60x performance boost} while offering more features.
        \item Reduced average code completion latency from \SI{16}{\ms} to
          \SI{1.09}{\ms} for LLVM codebase (3M Lines Of Code) and from
          \SI{119}{\ms} to \SI{1.9}{\ms} for Chromium codebase (16M LOC).
        \item Implemented Variable length Byte (VByte) compression algorithm
	  and reduced memory overhead by 60\% with no performance or precision
	  losses.
        \item Identified performance bottlenecks in LLVM infrastructure and made
	  commonly used serializer 3 times faster.
      \end{cvitems}
    }

  \cventry
    {Software Engineering Intern}
    {Google}
    {}
    {Jun 2016 - Sep 2016}
    {
      \begin{cvitems}
        \item Elevated the performance of
          \link{Clang-Rename}{https://clang.llvm.org/extra/clang-rename.html},
          Clang-based tool that can perform efficient renaming actions in
          large-scale projects such as renaming classes, functions, variables,
          arguments, namespaces etc.
        \item Fixed renaming bugs, added support for template classes,
	  introduced new functionality and gained users for Clang-Rename.
        \item Built Vim and Emacs plugins for Clang-Rename tool.
        \item Designed and prototyped
          \link{Clang-Refactor}{https://docs.google.com/document/d/1w9IkR0_Gqmd5w4CZ2t_ZDZrNLYVirQPyMS41533HQZE/edit?usp=sharing},
          which later transformed into \link{Clang's Refactoring
          Engine}{https://clang.llvm.org/docs/RefactoringEngine.html}.
        \item Started a discussion about building Language Server Protocol
          implementation which later resulted in clangd creation and became the
          main focus of the team.
        \item Added new checks to
          \link{Clang-Tidy}{https://clang.llvm.org/extra/clang-tidy/} and
          reduced false-positive rate for existing ones.
      \end{cvitems}
    }

  \cventry
    {Google Summer of Code Student}
    {Google}
    {}
    {Jun 2015 - Sep 2015}
    {
      \begin{cvitems}
        Implemented Code Clone Detection tool in Clang Static Analyzer and used
        it to detect over $400$ similar code pieces in Git, Vim, OpenSSL and
        other projects. The
        \link{overview}{https://github.com/kirillbobyrev/code-clone-detection-llvm-devmtg15-poster\#code-clone-detection-llvm-devmtg15-poster}
        of key results is hosted on GitHub.
      \end{cvitems}
    }

\end{cventries}
